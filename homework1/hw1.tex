%%%%%%%%%%%%%%%%%%%%%%%%%%%%%%%%%%%%%%%%%%%%%%%%%%%%%%%%%%%%%%%%%%%%%%%%
\documentclass[12pt]{article}
\usepackage{amssymb,amsthm}
\usepackage{amsmath,amssymb,CJK}
\usepackage{graphicx}
\usepackage{subfigure}
\usepackage{listings}
\usepackage{enumerate}

\openup 7pt\pagestyle{plain} \topmargin -40pt \textwidth
14.5cm\textheight 21.5cm
\parskip .09 truein
\baselineskip 4pt\lineskip 4pt \setcounter{page}{1}
\def\a{\alpha}\def\b{\beta}\def\d{\delta}\def\D{\Delta}\def\fs{\footnotesize}
\def\g{\gamma}
\def\G{\Gamma}\def\l{\lambda}\def\L{\Lambda}\def\o{\omiga}\def\p{\psi}
\def\se{\subseteq}\def\seq{\subseteq}\def\Si{\Sigma}\def\si{\sigma}\def\vp{\varphi}\def\es{\varepsilon}
\def\sc{\scriptstyle}\def\ssc{\scriptscriptstyle}\def\dis{\displaystyle}
\def\cl{\centerline}\def\ll{\leftline}\def\rl{\rightline}\def\nl{\newline}
\def\ol{\overline}\def\ul{\underline}\def\wt{\widetilde}\def\wh{\widehat}
\def\rar{\rightarrow}\def\Rar{\Rightarrow}\def\lar{\leftarrow}
\def\Lar{\Leftarrow}\def\Rla{\rightleftarrow}\def\bs{\backslash}
\def\ra{\rangle}\def\la{\langle}\def\hs{\hspace*}\def\rb{\raisebox}
\def\ni{\noindent}\def\hi{\hangindent}\def\ha{\hangafter}
\def\vs{\vspace*}
\def\hom#1{\mbox{\rm Hom($#1,#1$)}}
\def\thebeg{\vskip8pt\par\ni}
\def\theend{\vskip5pt\par}
\def\chabeg{\pagebreak\par}
\def\chaend{\vskip20pt\par}
\def\secbeg{\vskip15pt\par}
\def\secend{\vskip10pt\par}
\def\exebeg{\vskip10pt}
\def\exeend{\vskip6pt}
\def\undot#1{\mbox{$\mbox{#1}\hs{-1.5ex}_{_{\dis\centerdot}}\,\,$}}
\def\qed{\hfill\mbox{$\Box$}}
\def\C{\mathbb{C}}
\def\Q{\mathbb{Q}}
\def\ii{\mbox{\,{\bf i}\,}}
\def\jj{\mbox{\,{\bf j}\,}}
\def\AA{{\cal A}}
\def\BB{{\cal B}}
\def\DD{{\cal D}}
\def\EE{{\mbox{\bf 1}}}
\def\OO{{\mbox{\bf 0}}}
\def\kk{{\mbox{\bf k}}}
\def\R{\mathbb{R}}
\def\F{\mathbb{F}{\ssc\,}}
%\def\similar{\rb{-4pt}{\mbox{\,\~\,}}}
\def\similar{\backsim}
\def\Llra{\Longleftrightarrow}
\def\Lra{\Longrightarrow}
\def\Lla{\Longleftarrow}
\def\mat#1#2{\mbox{$\left(\begin{array}{#1}#2\end{array}\right)$}}
\def\det#1#2{\mbox{$\left|\begin{array}{#1}#2\end{array}\right|$}}
\def\eset{\emptyset}
\parindent=5ex
\setlength{\parindent}{0pt}
\setlength{\parskip}{1ex plus 0.5ex minus 0.2ex}
\newtheorem{Example}{\text{例}}
\begin{CJK*}{UTF8}{gbsn}

\date{}
\title{Homework 1}
\author{Qinglin Li, 5110309074}
\begin{document}
\maketitle
\section*{Problem 1}
\begin{enumerate}
\item
$\binom{10}{8}=\binom{10}{2}=\frac{10\times9}{2\times1}=45$
\item
$1^{-1}=1\mod{7}$\\
$2^{-1}=4\mod{7}$\\
$3^{-1}=5\mod{7}$\\
$4^{-1}=2\mod{7}$\\
$5^{-1}=3\mod{7}$\\
$6^{-1}=6\mod{7}$
\item
$449=2^0+2^6+2^7+2^8=4\times5^0+4\times5^1+2\times5^2+3\times5^3$\\
$137=2^0+2^3+2^7=2\times5^0+2\times5^1+1\times5^3$\\
$\binom{449}{137}=\binom{1}{1}\binom{0}{1}\binom{1}{0}\binom{1}{1}\binom{1}{0}=0\mod{2}$\\
$\binom{449}{137}=\binom{4}{2}\binom{4}{2}\binom{2}{0}\binom{3}{1}=3 \mod{5}$\\

By Chinese Remainder Theorem\\
$\binom{449}{137}=8 \mod10$
\end{enumerate}

\section*{Problem 2}
Firstly, choose $k(0\leq k \leq n)$ elements from $[n]$ as $A$. And then the remained $n-k$ elements can be either in $B$ or not.\\
$$\# \text{oredered pairs} = \sum_{k=0}^n \binom{n}{k}\cdot2^{n-k}=(1+2)^n=3^n$$

\section*{Problem 3}
For any $9\times9$ array, consider generating it by filling 1 to 81 into the units one by one.\\
Suppose every line or every column is filled for the first time after the number $k$ is filled.\\
WLOG, suppose every line is filled.\\
Because not all columns are filled before $k$ is filled, there must be at least one blank in all rows after $k$ is filled. In other word, there must be at least $9$ blanks adjacent to filled units.\\
Then the maxmium number filled into these blanks must be at least $k+9$ because $1$ to $k$ are used.\\
So the difference is at least $(k+9)-k=9$

\section*{Problem 4}
Suppose $\exists a, \exists b\neq c$, $\overline{ab}=\overline{ac}$, clearly $abc$ is a fun triple.\\
$\forall d\neq a,b,c$, either both $abd$ and $acd$ are fun triples, or none of $abd$ and $acd$ is fun triple.\\
Since the number of fun triples in $\{a,b,c,d\}$ is even, $bcd$ must be a fun triple.\\
So $\overline{bc}$ is of size $n$.\\

Otherwise $\forall a, \forall b\neq c, \overline{ab}\neq\overline{ac}$\\
Then there must exist a club $C$ such that $\exists a \not\in C$\\
For other people $b_1,b_2,\cdots,b_{n-1}, \forall i\neq j, \overline{ab_i}\neq \overline{ab_j}$\\
So $n$ clubs $\overline{ab_1},\cdots,\overline{ab_{n-1}},C$ are different.
\end{CJK*}
\end{document}